% CSCI3753 - Operating Systems
% Spring 2012
% Programming Assignment 4
% By Andy Sayler (3/6/12)

\documentclass[12pt]{article}

\usepackage[text={6.5in, 9in}, centering]{geometry}
\usepackage{graphicx}
\usepackage{url}

\title{Programming Assignment 3:\\Investigating the Linux Scheduler}
\author{
  CSCI 3753 - Operating Systems\\
  University of Colorado at Boulder\\
  Spring 2012\\
  By Andy Sayler and Junho Ahn and Richard Han
  Adopted from assignment by Dr. Alva Couch, Tufts University
}
\date{\emph{Due Date: Wednesday, April 18th, 2012 11:55pm}}

\begin{document}

\maketitle

\section{Assignment Introduction}

\section{Your Task}

\section{Some Implementation Ideas}

\section{What You Must Provide}

When you submit your assignment, you must provide the following as a
single archive file:
\begin{itemize}
\item A copy of all your code
\item A makefile that builds any necessary code
\item A README explaining how to build and run your code
\end{itemize}

\section{What's Included}

We provide some code to help get you started. Feel free to use it as a
jumping off point (appropriately cited).

\begin{enumerate}

\item {\bf pi.c} The source code for a statistically-based pi
  calculator. Accepts as the first argument the number of iterations to
  compute over. Example of a CPU bound process.

\item {\bf Makefile} A GNU Make makefile to build all the code listed
  here.

\item {\bf README} As the title so eloquently instructs: read it.

\end{enumerate}

\section{Extra Credit}

\section{Grading}

40\% of you grade will be based on the submission you provide.
To received full credit your submission must:
\begin{itemize}
\item Meet all requirements elicited in this document
\item Code must build with ``-Wall'' and ``-Wextra'' enabled,
  producing no errors or warnings.
\item Code must adhere to good coding practices.
\end{itemize}

The other 60\% of your grade will be determined via your grading
interview where you will be expected to explain your results and answer
questions regarding them and any concepts related to this assignment.

\section{Obtaining Code}
The starting code for this assignment is available on the Moodle and
on github. If you would like practice using a version control system,
consider forking the code from github. Using the github code is not
a requirement, but it will help to insure that you stay up to date
with any updates or changes to the supplied codebase. It is also
good practice for the kind of development one might expect to do in
a professional environment. And since your github code can be easily
shared, it can be a good way to show off your coding skills to
potential employers and other programmers.

Github code may be forked from the project page here:\\
\url{https://github.com/asayler/CU-CS3753-2012-PA4}.

\section{Resources}
Refer to your textbook and class notes on the Moodle for an overview
of OS paging policies and implementations.

The Internet\cite{tubes} is also a good resource for finding
information related to solving this assignment.

You may wish to consult the man pages for the following items, as they
will be useful and/or required to complete this assignment. Note that
the first argument to the ``man'' command is the chapter, insuring
that you access the appropriate version of each man page. See
\texttt{man man} for more information.

\begin{itemize}
\item \texttt{man 1 make}
\end{itemize}

\begin{thebibliography}{9}

\bibitem{K&R} Kernighan, Brian and Dennis, Ritchie.
  \newblock \emph{The C Programming Language}.
  \newblock Second Edition: 2009.
  \newblock Prentice Hall: New Jersey.

\bibitem{tubes} Stevens, Ted.
  \newblock \emph{Speech on Net Neutrality Bill}.
  \newblock 2006.
  \newblock \url{http://youtu.be/f99PcP0aFNE}.

\bibitem{StrunkWhite} Strunk, William, Jr. and White, E.B.
  \newblock \emph{The Elements of Style}.
  \newblock Fourth Edition: 2000.
  \newblock Pearson: New York.
  
\end{thebibliography}

\end{document}  
